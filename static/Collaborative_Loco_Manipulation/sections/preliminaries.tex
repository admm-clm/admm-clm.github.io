\section{Preliminaries}
\label{sec:preliminaries}
\subsection{Consensus ADMM}
We adopt the standard \textit{consensus} ADMM formulation \cite{boyd2011distributed} for problems of the form
\begin{equation}\label{eq:consensusADMM}
\begin{aligned}
    \underset{\Bar{\v x},\{\v x_i\}_{i=1}^{N}}{\text{min}} \quad
    & \sum_{i = 1}^{N} f_i(\v x_i) + g(\Bar{\v x}) \\
    \text{s.t.} \quad
    & \v x_i = \Bar{\v x}, \quad i = 1,\dots,N ,
\end{aligned}
\end{equation}
where $\v x_i$ denotes the local decision variables of subsystem $i$, $\Bar{\v x}$ is a global consensus variable, and $g(\Bar{\v x})$ is an optional regularization term. 
By establishing the scaled augmented Lagrangian (AL) of~\eqref{eq:consensusADMM},
consensus ADMM proceeds by alternating updates of the local variables, the global consensus variable, and the dual variables:
\begin{subequations}\label{eq:consensus-updates}
\begin{align}
    \v x_i^{k+1}
    &:= \arg\min_{\v x_i}
    \left(
        f_i(\v x_i)
        + \frac{\rho}{2}
        \|\v x_i - \Bar{\v x}^k + \v w_i^k\|^2
    \right),
    \quad \forall i, \label{eq:x-update} \\[4pt]
    \Bar{\v x}^{k+1}
    &:= \arg\min_{\Bar{\v x}}
    \left(
        g(\Bar{\v x})
        + \frac{\rho}{2}
        \sum_{i=1}^{N}
        \|\v x_i^{k+1} - \Bar{\v x} + \v w_i^k\|^2
    \right), \label{eq:z-update} \\[4pt]
    \v w_i^{k+1}
    &:= \v w_i^k + \v x_i^{k+1} - \Bar{\v x}^{k+1},
    \quad \forall i. \label{eq:dual-update}
\end{align}
\end{subequations}
where $\{\v w_i\}$ are the scaled dual variables and $\rho > 0$ is the penalty parameter.
This structure enables parallel optimization of the local subproblems~\eqref{eq:x-update}, followed by a lightweight consensus update~\eqref{eq:z-update}. The updates in~\eqref{eq:consensus-updates} correspond to standard \textit{consensus} ADMM. 
We use the standard \textit{Gauss–Seidel} consensus variant, which provides better convergence than the fully parallel \textit{Jacobi} variant while keeping the consensus subproblem inexpensive.
% The \textit{Jacobi} ADMM updates all primal variables, including the consensus variable $\Bar{\v x}$, in parallel using values from the previous iteration, i.e., the consensus step uses ${\v x_i^k}$ instead of ${\v x_i^{k+1}}$.

\subsection{Centralized Optimization Formulation}
The centralized optimization formulation is written as:
\begin{subequations} 
\begin{align}
	\underset{\v U, \v X}{\text{min}} \ \mathcal{C}(\v U, \v X) \coloneqq & \sum_{k = 0}^{N-1}l_k(\v x[k], \v u[k]) + l_N(\v x[N]) \\
	\text{s.t.} \quad & \v x[k+1] = \mathcal{D}(\v x[k], \v u[k]) \\
    & \v x[0] = \v x_{\text{init}} \\
    & \v g(\v x[k], \v u[k]) = 0 \\
    & \v h(\v x[k], \v u[k]) \leq 0
\end{align}
\end{subequations}
where $\v x[k]$ and $\v u[k]$ denote the state and control variables at the $k$-th time step over a horizon of $N+1$ steps, with initial condition $\v x_{\text{init}}$. The stacked state and control trajectories are defined as $\v X := [\v x[0]^{\top}, \v x[1]^{\top}, \dots,\v x[N]^{\top}]^{\top}$ and $\v U := [\v u[0]^{\top}, \v u[1]^{\top}, \dots,\v u[N-1]^{\top}]^{\top}$. At each time step, the state and control vectors are decomposed into payload and robot components: $\v x[k] := [\v x_{0}[k]^{\top}, \dots, \v x_{i}[k]^{\top}, \dots,\v x_{R}[k]^{\top}]^{\top}$, $\v u[k] := [\v u_{1}[k]^{\top}, \dots, \v u_{i}[k]^{\top}, \dots,\v u_{R}[k]^{\top}]^{\top}$,
% \begin{equation}
% \begin{aligned}\nonumber
%     \v x[k] := [\v x_{0}[k]^{\top}, \dots, \v x_{i}[k]^{\top}, \dots,\v x_{R}[k]^{\top}]^{\top},\\
%     \v u[k] := [\v u_{1}[k]^{\top}, \dots, \v u_{i}[k]^{\top}, \dots,\v u_{R}[k]^{\top}]^{\top}
% \end{aligned}
% \end{equation}
where index $i \in \{0,1,\dots,R\}$ denotes the rigid body, with $i=0$ corresponding to the payload and $i>0$ to the robots. The payload has no direct control input, as its motion is governed by the interaction forces and torques exerted by the robots. The payload state is defined as
\begin{equation}
    \begin{aligned}
        \v x_{0}[k] := [\v r_{0}[k]^{\top}, \dot{\v r}_{0}[k]^{\top}, \vg \theta_{0}[k]^{\top}, \v l_{0}[k]^{\top}]^{\top}
    \end{aligned}
\end{equation}
where $\v r$, $\dot{\v r}$, $\vg \theta$, and $\v l \in \mathbb{R}^3$ denote the position, linear velocity, Euler angles, and angular momentum of the rigid body, respectively. In this work, each robot is also approximated as a single rigid body to reduce optimization complexity; however, the proposed distributed framework readily extends to full-order models with joint-level dynamics and whole-body constraints. For each robot $i \in \{1,\dots,R\}$, the state and control variables are currently defined as
\begin{equation}
    \begin{aligned}
        & \forall j \in \{0, \dots, n_f-1\},\\
        & \v x_{i}[k] := [\v r_{i}[k]^{\top}, \dot{\v r}_{i}[k]^{\top}, \vg \theta_{i}[k]^{\top}, \v l_{i}[k]^{\top}, \v p_{i,j}[k]^{\top}]^{\top}\\
        & \v u_{i}[k] := [\v f_{i,j}[k]^{\top}, \dot{\v p}_{i,j}[k]^{\top}, \v f_{i,h}[k]^{\top}, \vg \tau_{i,h}[k]^{\top}]^{\top}
    \end{aligned}
\end{equation}
where $\v p_{i,j}$ and $\v f_{i,j}$ denote the position and contact force of the $j$-th foot, $n_f$ is the number of feet, and $\v f_{i,h}$ and $\vg \tau_{i,h}$ represent the manipulation force and torque applied on the robot’s arm end-effector (EE) when grasping the payload.

The coupled system dynamics $\mathcal{D}$ can be decomposed into payload and robot components $\mathcal{D}^0$ and $\mathcal{D}^{i}$. For clarity of presentation, the time-step superscript $k$ is omitted in the following equations. The second-order dynamics of the payload are given by
\begin{subequations}\label{eq:raw_payload_dynamics}
    \begin{align}
        \ddot{\v r}_{0} &= \frac{1}{m_0}\!\left(- \sum\nolimits_i \v f_{i,h}\right) + \boldsymbol{g}, \\
        \dot{\v l}_0 &= \sum\nolimits_i 
        \big(\v p_{i,h}(\v r_0,\vg \theta_0) - \v r_0\big) \times (-\v f_{i,h})
        - \vg \tau_{i,h},
    \end{align}
\end{subequations}
where $\v p_{i,h}(\cdot,\cdot)$ denotes the position of the $i$-th robot’s arm EE expressed in the inertial frame, obtained by transforming a constant handle offset in the cargo frame. The second-order dynamics of the $i$-th robot are expressed as
\begin{subequations}\label{eq:raw_robot_dynamics}
    \begin{align}
        \ddot{\v r}_{i} =&
        \frac{1}{m_i}\!\left(\sum\nolimits_{j} \v f_{i,j} + \v f_{i,h}\right) + \boldsymbol{g}, \\
        \dot{\v l}_i =&
        \sum\nolimits_j (\v p_{i,j} - \v r_i) \times \v f_{i,j}
        + \\ & (\v p_{i,h}(\v r_0,\vg \theta_0) - \v r_i) \times \v f_{i,h}
        + \vg \tau_{i,h},
    \end{align}
\end{subequations}
% where $\v p_{i,j}$ and $\v f_{i,j}$ denote the position and contact force of the $j$-th foot of robot $i$, respectively.

All continuous-time dynamics are discretized using the backward Euler method for use in the optimal control formulation. The equality and inequality constraint functions $\v g(\cdot)$ and $\v h(\cdot)$ are introduced later.