\section{Related Work}
\label{sec:related_work}
\subsection{Collaborative Transportation and Manipulation}
Multi-robot collaborative transportation has been studied on aerial and ground platforms, with approaches broadly categorized as centralized, decentralized, or leader–follower. Centralized methods scale poorly \cite{nikou2017nonlinear,sun2025agile}, while decentralized approaches trade coordination optimality for scalability \cite{khatib1996coordination,verginis2018communication,culbertson2018decentralized}. Leader–follower strategies \cite{farivarnejad2022multirobot,sugar2002control} avoid explicit group-level trajectory optimization.
% However, optimal trajectory planning for the robot group is not addressed directly in leader-follower approaches.
In contrast, collaborative loco-manipulation with legged robots remains less explored due to high DoFs and complex dynamics, with different paradigms arising from how robots interact with the payload.

\textbf{Holonomically constrained systems},
% where robots are mechanically linked via cables or rigid connections, have been addressed using decentralized learning-based controllers relying only on local state and relative payload information \cite{pandit2024} and hierarchical architectures combining centralized or distributed planners with decentralized controllers \cite{kim2023,yang2022}. While these methods scale well, their applicability is limited by restrictive mechanical assumptions and reduced interaction flexibility.
using cables or rigid links, enable scalable decentralized control through learning-based or hierarchical architectures \cite{pandit2024,kim2023,yang2022}, but their applicability is limited by restrictive mechanical assumptions and reduced interaction flexibility.
% \textbf{Prehensile collaborative loco-manipulation}, which enables more flexible interactions through grasping or holding, has primarily been studied using centralized or hierarchical approaches, including centralized MPC with simplified rigid-body models \cite{devincenti2023}, and hierarchical framework combining payload planning and decentralized whole-body control for collision-free transport of large objects \cite{rigo2025hierarchical}. 
% Passive-arm designs have also been used to enable leader–follower coordination through intrinsic mechanical impedance \cite{turrisi2024}. Learning-based decentralized approaches have recently demonstrated contact-only lifting and transport without communication by inducing rigid-like coordination through reward shaping \cite{pandit2025multi}, and hierarchical RL has been applied to collaborative pick-and-place tasks \cite{an2025collaborative}. Despite their flexibility, these methods often face scalability limitations due to centralized optimization, strong hierarchical dependencies, or complex reward and curriculum design.
\textbf{Prehensile collaborative loco-manipulation} allows more flexible grasp-based interaction and has been addressed using centralized MPC with simplified models \cite{devincenti2023}, hierarchical planning frameworks \cite{rigo2025hierarchical}, passive-arm leader–follower designs \cite{turrisi2024}, and learning-based decentralized strategies \cite{pandit2025multi,an2025collaborative}. However, many of these approaches depend on centralized optimization that limits scalability, strong hierarchy, or complex reward design.
% In contrast, \textbf{non-prehensile approaches} typically rely on indirect manipulation such as pushing \cite{feng2025learning, sombolestan2024} or carrying \cite{ji2021reinforcement} without extra mechanical design. Long-horizon pushing \cite{feng2025learning, sombolestan2024} tasks have been achieved by hierarchical planning frameworks that combine high-level safety-critical payload planning with decentralized RL or MPC controller to handle contact uncertainty and obstacles. These methods support long-horizon tasks but lack direct force control, limiting manipulation precision and task diversity.
\textbf{Non-prehensile approaches} rely on indirect manipulation such as pushing or carrying \cite{feng2025learning,sombolestan2024,ji2021reinforcement}, typically within hierarchical planning frameworks. While suitable for long-horizon tasks, they lack direct force control and manipulation precision.

Motivated by the flexibility of prehensile manipulation and the scalability limitations of existing centralized approaches, we focus on prehensile collaborative loco-manipulation and propose a distributed MPC framework based on alternating optimization, which enables scalable subsystem updates with local communication while retaining much of the centralized solution quality and improving computational scalability.


\subsection{ADMM for Multi-Robot Distributed Control}
Many OCPs exhibit intrinsic distributed structure despite coupling effects \cite{zhao2024survey}. In multi-robot systems, spatial separability enables decomposition via alternating optimization, particularly ADMM \cite{boyd2011distributed}. Multi-robot path finding (MAPF) is an example where these ideas have been successfully applied \cite{saravanos2023distributed,tajbakhsh2025asynchronous}: robots maintain local copies of their own and neighboring agents' trajectories, enforce consensus over coupling constraints such as collision avoidance, and achieve scalability to large teams. However, these formulations mostly consider simplified robot models, which keep subproblem computationally tractable despite the augmented state dimension for each robot.

Collaborative manipulation also exhibits an implicit distributed structure, although the subsystems are coupled through shared object dynamics. 
To avoid the growing number of control inputs for the robot-object subsystem, \cite{shorinwa2020scalable} proposes a decomposition in which each subsystem optimizes only the force and torque contributed by a single robot, while consensus is enforced on the object trajectory. A similar idea is extended to contact-implicit manipulation settings in \cite{shorinwa_disco_2024}. 
In contrast, collaborative loco-manipulation with legged manipulators involves high-dimensional, multi-contact dynamics. Rather than augmenting robot states with object states, we treat the object as an independent subsystem and couple subsystems only through interaction forces and torques under consensus. This keeps subproblems compact and enables scalable distributed MPC despite nonlinear, high-dimensional dynamics.
% This formulation further reduces the size of each subproblem and is critical for enabling scalable distributed MPC in the presence of strong nonlinearity and high-dimensional robot dynamics.