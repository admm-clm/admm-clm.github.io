%%%%%%%%%%%%%%%%%%%%%%%%%%%%%%%%%%%%%%%%%%%%%%%%%%%%%%%%%%%%%%%%%%%%%%%%%%%%%%%%
%2345678901234567890123456789012345678901234567890123456789012345678901234567890
%        1         2         3         4         5         6         7         8

\documentclass[letterpaper, 10 pt, conference]{ieeeconf}  % Comment this line out if you need a4paper

%\documentclass[a4paper, 10pt, conference]{ieeeconf}      % Use this line for a4 paper

\IEEEoverridecommandlockouts

\usepackage{amsmath}
\usepackage{cite}
\usepackage{amssymb,amsfonts}
\usepackage[ruled,vlined]{algorithm2e}
\usepackage{graphicx}
\usepackage{subfloat}
\usepackage{textcomp}
\usepackage{xcolor}
% \usepackage{algorithm}
\usepackage{verbatim} 
\usepackage{mathrsfs}
\usepackage{subfigure}
\usepackage{empheq}
\usepackage{courier}
\usepackage{lipsum}
\usepackage{comment}
\usepackage{gensymb}
\usepackage{graphicx}
\usepackage{mathtools, cuted}
\usepackage{bm}
\usepackage{ wasysym } % For /ocircle
\usepackage{soul}
\usepackage[colorlinks,citecolor=red,urlcolor=black,bookmarks=false,hypertexnames=true,hidelinks]{hyperref}

\let\proof\relax 
\let\endproof\relax
\usepackage{amsthm}
\newtheoremstyle{remarkcolon}
  {0.5\baselineskip} % Space above
  {0.5\baselineskip} % Space below
  {\itshape}         % <-- Body font (italic)
  {0pt}              % Indent
  {\bfseries}        % <-- Head font (bold, non-italic)
  {:}                % Punctuation after head
  {0.5em}            % Space after head
  {}                 % Head spec
\theoremstyle{remarkcolon}
\newtheorem{remark}{Remark}

\newcommand{\vg}[1]{\bm{#1}}
\renewcommand{\v}[1]{\mathbf{#1}}

\title{\LARGE \bf
ACLM: ADMM-Based Distributed Model Predictive Control for Collaborative Loco-Manipulation
}

\author{\large Anonymous Author(s)}

% \author{Albert Author$^{1}$ and Bernard D. Researcher$^{2}$% <-this % stops a space
% \thanks{*This work was not supportFaculty of Electrical Engineering, Mathematics and Computer Science,ed by any organization}% <-this % stops a space
% \thanks{$^{1}$Albert Author is with 
%         University of Twente, 7500 AE Enschede, The Netherlands
%         {\tt\small albert.author@papercept.net}}%
% \thanks{$^{2}$Bernard D. Researcheris with the Department of Electrical Engineering, Wright State University,
%         Dayton, OH 45435, USA
%         {\tt\small b.d.researcher@ieee.org}}%
% }


\begin{document}



\maketitle
\thispagestyle{empty}
\pagestyle{empty}


%%%%%%%%%%%%%%%%%%%%%%%%%%%%%%%%%%%%%%%%%%%%%%%%%%%%%%%%%%%%%%%%%%%%%%%%%%%%%%%%
\begin{abstract}
Collaborative transportation of heavy payloads via loco-manipulation is a challenging yet essential capability for legged robots operating in complex, unstructured environments. Centralized planning methods, e.g., holistic trajectory optimization, capture dynamic coupling among robots and payloads but scale poorly with system size, limiting real-time applicability. In contrast, hierarchical and fully decentralized approaches often neglect force and dynamic interactions, leading to conservative behavior. This study proposes an Alternating Direction Method of Multipliers (ADMM)-based distributed model predictive control framework for collaborative loco-manipulation with a team of quadruped robots with manipulators. By exploiting the payload-induced coupling structure, the global optimal control problem is decomposed into parallel individual-robot-level subproblems with consensus constraints. The distributed planner operates in a receding-horizon fashion and achieves fast convergence, requiring only a few ADMM iterations per planning cycle. A wrench-aware whole-body controller executes the planned trajectories, tracking both motion and interaction wrenches. Extensive simulations with up to four robots demonstrate scalability, real-time performance, and robustness to model uncertainty.

\end{abstract}

\section{Introduction}
\label{sec:introduction}
\begin{figure}
    \centering
    \includegraphics[width=0.94\linewidth]{figures/flowchart.pdf}
    \vspace{-1mm}
    \caption{ADMM-based distributed MPC framework. The robot subproblems are solved in parallel with consensus on interaction wrenches, and the resulting trajectories are executed by local WBC.}
    \label{fig:system_diagram}
    \vspace{-5mm}
\end{figure}

Recent robotics research has increasingly focused on enhancing loco-manipulation capabilities, aiming to automate repetitive and labor-intensive tasks while preserving all-terrain mobility. Among these tasks, collaborative transportation of heavy and oversized loads via loco-manipulation has attracted significant attention due to their prevalence in logistics, mining, construction, agriculture, and search-and-rescue operations \cite{farivarnejad2022multirobot}. 
Collaborative loco-manipulation, in which multiple legged robots equipped with manipulators cooperatively transport a shared load, offers a promising approach by distributing both the payload weight and the control effort across the robot team.

To simplify coordination, many approaches adopt hierarchical frameworks that decouple payload and robot planning \cite{rigo2025hierarchical,yang2022}. While scalable, these designs often rely on quasi-static assumptions or neglect dynamic and kinematic coupling between robots and the payload, leading to conservative motions. For quadrupedal manipulators, where locomotion and manipulation forces are strongly coupled, ignoring these interactions can degrade performance and stability especially on rough terrains (see Fig.~\ref{fig:system_diagram}), motivating planning frameworks that explicitly model force and dynamic coupling.

In contrast, centralized planning frameworks can naturally account for dynamic coupling and shared constraints among the robots and the payload by solving a unified optimal control problem (OCP) \cite{devincenti2023,sun2025agile}. However, this comes at the cost of significantly increased computational complexity. Compared to teams of wheeled mobile manipulators or quadrotors, cooperative quadruped manipulators present additional challenges due to high number of degrees of freedom (DoFs), hybrid contact dynamics, and rough terrain. As a result, the scalability of fully centralized planning with respect to the number of robots becomes a critical bottleneck, limiting its applicability for real-time replanning and control.

In this work, we explicitly analyze the coupling structure between quadruped manipulators and a shared payload and leverage the Alternating Direction Method of Multipliers (ADMM) \cite{boyd2011distributed} to decompose the resulting multi-robot optimal control problem into tractable subproblems. Although the system exhibits strong coupling through the payload dynamics, each robot interacts directly only with the payload rather than with other robots. This star-shaped coupling structure enables parallel optimization across robots through carefully designed consensus constraints, allowing distributed computation while preserving the essential force and dynamic interactions induced by the shared load.

We further adopt a real-time, receding-horizon implementation of the distributed planner within a model predictive control (MPC) framework. Notably, the solution from the previous MPC window provides an effective warm start for the current optimization, allowing the ADMM solver to converge to a satisfactory consensus with only a small number of ADMM iterations per planning cycle. At a higher control rate, a wrench-aware whole-body controller (WBC) tracks the planned end-effector poses, foot placements, and interaction wrenches, bridging distributed planning and execution. Although whole-body inverse dynamics is well studied \cite{bellicoso2019alma}, to our knowledge this is the first integrated distributed MPC–WBC pipeline that accounts for coupling effects in prehensile collaborative loco-manipulation with multiple legged manipulators in challenging environments.

Our contributions can be summarized as follows:
\begin{itemize}
    \item We propose an ADMM-based distributed MPC framework that decomposes a tightly coupled optimal control problem involving multiple quadruped manipulators and a shared payload into tractable subproblems.
    \item We fully integrate a wrench-aware WBC with the distributed MPC, which tracks the optimized motion and interaction trajectories, including end-effector poses, foot contacts, and manipulation wrench.
    \item We evaluate teams of 2–4 robots on diverse collaborative transportation tasks, including obstacle avoidance and rough-terrain traversal, demonstrating real-time performance independent of team size (50 Hz with perceptive inputs and 100 Hz without) and robustness to model uncertainties.
\end{itemize}


\section{Related Work}
\label{sec:related_work}
\subsection{Collaborative Transportation and Manipulation}
Multi-robot collaborative transportation has been studied on aerial and ground platforms, with approaches broadly categorized as centralized, decentralized, or leader–follower. Centralized methods scale poorly \cite{nikou2017nonlinear,sun2025agile}, while decentralized approaches trade coordination optimality for scalability \cite{khatib1996coordination,verginis2018communication,culbertson2018decentralized}. Leader–follower strategies \cite{farivarnejad2022multirobot,sugar2002control} avoid explicit group-level trajectory optimization.
% However, optimal trajectory planning for the robot group is not addressed directly in leader-follower approaches.
In contrast, collaborative loco-manipulation with legged robots remains less explored due to high DoFs and complex dynamics, with different paradigms arising from how robots interact with the payload.

\textbf{Holonomically constrained systems},
% where robots are mechanically linked via cables or rigid connections, have been addressed using decentralized learning-based controllers relying only on local state and relative payload information \cite{pandit2024} and hierarchical architectures combining centralized or distributed planners with decentralized controllers \cite{kim2023,yang2022}. While these methods scale well, their applicability is limited by restrictive mechanical assumptions and reduced interaction flexibility.
using cables or rigid links, enable scalable decentralized control through learning-based or hierarchical architectures \cite{pandit2024,kim2023,yang2022}, but their applicability is limited by restrictive mechanical assumptions and reduced interaction flexibility.
% \textbf{Prehensile collaborative loco-manipulation}, which enables more flexible interactions through grasping or holding, has primarily been studied using centralized or hierarchical approaches, including centralized MPC with simplified rigid-body models \cite{devincenti2023}, and hierarchical framework combining payload planning and decentralized whole-body control for collision-free transport of large objects \cite{rigo2025hierarchical}. 
% Passive-arm designs have also been used to enable leader–follower coordination through intrinsic mechanical impedance \cite{turrisi2024}. Learning-based decentralized approaches have recently demonstrated contact-only lifting and transport without communication by inducing rigid-like coordination through reward shaping \cite{pandit2025multi}, and hierarchical RL has been applied to collaborative pick-and-place tasks \cite{an2025collaborative}. Despite their flexibility, these methods often face scalability limitations due to centralized optimization, strong hierarchical dependencies, or complex reward and curriculum design.
\textbf{Prehensile collaborative loco-manipulation} allows more flexible grasp-based interaction and has been addressed using centralized MPC with simplified models \cite{devincenti2023}, hierarchical planning frameworks \cite{rigo2025hierarchical}, passive-arm leader–follower designs \cite{turrisi2024}, and learning-based decentralized strategies \cite{pandit2025multi,an2025collaborative}. However, many of these approaches depend on centralized optimization that limits scalability, strong hierarchy, or complex reward design.
% In contrast, \textbf{non-prehensile approaches} typically rely on indirect manipulation such as pushing \cite{feng2025learning, sombolestan2024} or carrying \cite{ji2021reinforcement} without extra mechanical design. Long-horizon pushing \cite{feng2025learning, sombolestan2024} tasks have been achieved by hierarchical planning frameworks that combine high-level safety-critical payload planning with decentralized RL or MPC controller to handle contact uncertainty and obstacles. These methods support long-horizon tasks but lack direct force control, limiting manipulation precision and task diversity.
\textbf{Non-prehensile approaches} rely on indirect manipulation such as pushing or carrying \cite{feng2025learning,sombolestan2024,ji2021reinforcement}, typically within hierarchical planning frameworks. While suitable for long-horizon tasks, they lack direct force control and manipulation precision.

Motivated by the flexibility of prehensile manipulation and the scalability limitations of existing centralized approaches, we focus on prehensile collaborative loco-manipulation and propose a distributed MPC framework based on alternating optimization, which enables scalable subsystem updates with local communication while retaining much of the centralized solution quality and improving computational scalability.


\subsection{ADMM for Multi-Robot Distributed Control}
Many OCPs exhibit intrinsic distributed structure despite coupling effects \cite{zhao2024survey}. In multi-robot systems, spatial separability enables decomposition via alternating optimization, particularly ADMM \cite{boyd2011distributed}. Multi-robot path finding (MAPF) is an example where these ideas have been successfully applied \cite{saravanos2023distributed,tajbakhsh2025asynchronous}: robots maintain local copies of their own and neighboring agents' trajectories, enforce consensus over coupling constraints such as collision avoidance, and achieve scalability to large teams. However, these formulations mostly consider simplified robot models, which keep subproblem computationally tractable despite the augmented state dimension for each robot.

Collaborative manipulation also exhibits an implicit distributed structure, although the subsystems are coupled through shared object dynamics. 
To avoid the growing number of control inputs for the robot-object subsystem, \cite{shorinwa2020scalable} proposes a decomposition in which each subsystem optimizes only the force and torque contributed by a single robot, while consensus is enforced on the object trajectory. A similar idea is extended to contact-implicit manipulation settings in \cite{shorinwa_disco_2024}. 
In contrast, collaborative loco-manipulation with legged manipulators involves high-dimensional, multi-contact dynamics. Rather than augmenting robot states with object states, we treat the object as an independent subsystem and couple subsystems only through interaction forces and torques under consensus. This keeps subproblems compact and enables scalable distributed MPC despite nonlinear, high-dimensional dynamics.
% This formulation further reduces the size of each subproblem and is critical for enabling scalable distributed MPC in the presence of strong nonlinearity and high-dimensional robot dynamics.

\section{Preliminaries}
\label{sec:preliminaries}
\subsection{Consensus ADMM}
We adopt the standard \textit{consensus} ADMM formulation \cite{boyd2011distributed} for problems of the form
\begin{equation}\label{eq:consensusADMM}
\begin{aligned}
    \underset{\Bar{\v x},\{\v x_i\}_{i=1}^{N}}{\text{min}} \quad
    & \sum_{i = 1}^{N} f_i(\v x_i) + g(\Bar{\v x}) \\
    \text{s.t.} \quad
    & \v x_i = \Bar{\v x}, \quad i = 1,\dots,N ,
\end{aligned}
\end{equation}
where $\v x_i$ denotes the local decision variables of subsystem $i$, $\Bar{\v x}$ is a global consensus variable, and $g(\Bar{\v x})$ is an optional regularization term. 
By establishing the scaled augmented Lagrangian (AL) of~\eqref{eq:consensusADMM},
consensus ADMM proceeds by alternating updates of the local variables, the global consensus variable, and the dual variables:
\begin{subequations}\label{eq:consensus-updates}
\begin{align}
    \v x_i^{k+1}
    &:= \arg\min_{\v x_i}
    \left(
        f_i(\v x_i)
        + \frac{\rho}{2}
        \|\v x_i - \Bar{\v x}^k + \v w_i^k\|^2
    \right),
    \quad \forall i, \label{eq:x-update} \\[4pt]
    \Bar{\v x}^{k+1}
    &:= \arg\min_{\Bar{\v x}}
    \left(
        g(\Bar{\v x})
        + \frac{\rho}{2}
        \sum_{i=1}^{N}
        \|\v x_i^{k+1} - \Bar{\v x} + \v w_i^k\|^2
    \right), \label{eq:z-update} \\[4pt]
    \v w_i^{k+1}
    &:= \v w_i^k + \v x_i^{k+1} - \Bar{\v x}^{k+1},
    \quad \forall i. \label{eq:dual-update}
\end{align}
\end{subequations}
where $\{\v w_i\}$ are the scaled dual variables and $\rho > 0$ is the penalty parameter.
This structure enables parallel optimization of the local subproblems~\eqref{eq:x-update}, followed by a lightweight consensus update~\eqref{eq:z-update}. The updates in~\eqref{eq:consensus-updates} correspond to standard \textit{consensus} ADMM. 
We use the standard \textit{Gauss–Seidel} consensus variant, which provides better convergence than the fully parallel \textit{Jacobi} variant while keeping the consensus subproblem inexpensive.
% The \textit{Jacobi} ADMM updates all primal variables, including the consensus variable $\Bar{\v x}$, in parallel using values from the previous iteration, i.e., the consensus step uses ${\v x_i^k}$ instead of ${\v x_i^{k+1}}$.

\subsection{Centralized Optimization Formulation}
The centralized optimization formulation is written as:
\begin{subequations} 
\begin{align}
	\underset{\v U, \v X}{\text{min}} \ \mathcal{C}(\v U, \v X) \coloneqq & \sum_{k = 0}^{N-1}l_k(\v x[k], \v u[k]) + l_N(\v x[N]) \\
	\text{s.t.} \quad & \v x[k+1] = \mathcal{D}(\v x[k], \v u[k]) \\
    & \v x[0] = \v x_{\text{init}} \\
    & \v g(\v x[k], \v u[k]) = 0 \\
    & \v h(\v x[k], \v u[k]) \leq 0
\end{align}
\end{subequations}
where $\v x[k]$ and $\v u[k]$ denote the state and control variables at the $k$-th time step over a horizon of $N+1$ steps, with initial condition $\v x_{\text{init}}$. The stacked state and control trajectories are defined as $\v X := [\v x[0]^{\top}, \v x[1]^{\top}, \dots,\v x[N]^{\top}]^{\top}$ and $\v U := [\v u[0]^{\top}, \v u[1]^{\top}, \dots,\v u[N-1]^{\top}]^{\top}$. At each time step, the state and control vectors are decomposed into payload and robot components: $\v x[k] := [\v x_{0}[k]^{\top}, \dots, \v x_{i}[k]^{\top}, \dots,\v x_{R}[k]^{\top}]^{\top}$, $\v u[k] := [\v u_{1}[k]^{\top}, \dots, \v u_{i}[k]^{\top}, \dots,\v u_{R}[k]^{\top}]^{\top}$,
% \begin{equation}
% \begin{aligned}\nonumber
%     \v x[k] := [\v x_{0}[k]^{\top}, \dots, \v x_{i}[k]^{\top}, \dots,\v x_{R}[k]^{\top}]^{\top},\\
%     \v u[k] := [\v u_{1}[k]^{\top}, \dots, \v u_{i}[k]^{\top}, \dots,\v u_{R}[k]^{\top}]^{\top}
% \end{aligned}
% \end{equation}
where index $i \in \{0,1,\dots,R\}$ denotes the rigid body, with $i=0$ corresponding to the payload and $i>0$ to the robots. The payload has no direct control input, as its motion is governed by the interaction forces and torques exerted by the robots. The payload state is defined as
\begin{equation}
    \begin{aligned}
        \v x_{0}[k] := [\v r_{0}[k]^{\top}, \dot{\v r}_{0}[k]^{\top}, \vg \theta_{0}[k]^{\top}, \v l_{0}[k]^{\top}]^{\top}
    \end{aligned}
\end{equation}
where $\v r$, $\dot{\v r}$, $\vg \theta$, and $\v l \in \mathbb{R}^3$ denote the position, linear velocity, Euler angles, and angular momentum of the rigid body, respectively. In this work, each robot is also approximated as a single rigid body to reduce optimization complexity; however, the proposed distributed framework readily extends to full-order models with joint-level dynamics and whole-body constraints. For each robot $i \in \{1,\dots,R\}$, the state and control variables are currently defined as
\begin{equation}
    \begin{aligned}
        & \forall j \in \{0, \dots, n_f-1\},\\
        & \v x_{i}[k] := [\v r_{i}[k]^{\top}, \dot{\v r}_{i}[k]^{\top}, \vg \theta_{i}[k]^{\top}, \v l_{i}[k]^{\top}, \v p_{i,j}[k]^{\top}]^{\top}\\
        & \v u_{i}[k] := [\v f_{i,j}[k]^{\top}, \dot{\v p}_{i,j}[k]^{\top}, \v f_{i,h}[k]^{\top}, \vg \tau_{i,h}[k]^{\top}]^{\top}
    \end{aligned}
\end{equation}
where $\v p_{i,j}$ and $\v f_{i,j}$ denote the position and contact force of the $j$-th foot, $n_f$ is the number of feet, and $\v f_{i,h}$ and $\vg \tau_{i,h}$ represent the manipulation force and torque applied on the robot’s arm end-effector (EE) when grasping the payload.

The coupled system dynamics $\mathcal{D}$ can be decomposed into payload and robot components $\mathcal{D}^0$ and $\mathcal{D}^{i}$. For clarity of presentation, the time-step superscript $k$ is omitted in the following equations. The second-order dynamics of the payload are given by
\begin{subequations}\label{eq:raw_payload_dynamics}
    \begin{align}
        \ddot{\v r}_{0} &= \frac{1}{m_0}\!\left(- \sum\nolimits_i \v f_{i,h}\right) + \boldsymbol{g}, \\
        \dot{\v l}_0 &= \sum\nolimits_i 
        \big(\v p_{i,h}(\v r_0,\vg \theta_0) - \v r_0\big) \times (-\v f_{i,h})
        - \vg \tau_{i,h},
    \end{align}
\end{subequations}
where $\v p_{i,h}(\cdot,\cdot)$ denotes the position of the $i$-th robot’s arm EE expressed in the inertial frame, obtained by transforming a constant handle offset in the cargo frame. The second-order dynamics of the $i$-th robot are expressed as
\begin{subequations}\label{eq:raw_robot_dynamics}
    \begin{align}
        \ddot{\v r}_{i} =&
        \frac{1}{m_i}\!\left(\sum\nolimits_{j} \v f_{i,j} + \v f_{i,h}\right) + \boldsymbol{g}, \\
        \dot{\v l}_i =&
        \sum\nolimits_j (\v p_{i,j} - \v r_i) \times \v f_{i,j}
        + \\ & (\v p_{i,h}(\v r_0,\vg \theta_0) - \v r_i) \times \v f_{i,h}
        + \vg \tau_{i,h},
    \end{align}
\end{subequations}
% where $\v p_{i,j}$ and $\v f_{i,j}$ denote the position and contact force of the $j$-th foot of robot $i$, respectively.

All continuous-time dynamics are discretized using the backward Euler method for use in the optimal control formulation. The equality and inequality constraint functions $\v g(\cdot)$ and $\v h(\cdot)$ are introduced later.

\input{sections/approach}

\section{Results}
\label{sec:experiment}
\subsection{Experimental Setup}
Unless otherwise specified, we use at most 2 ADMM iterations and 1 SQP iteration per MPC solve. The rationale is discussed in Sec.~\ref{subsec:scalability_convergence}. We refer to different iteration limits as ADMM-SQP configurations. The consensus constraint tolerance is set to $5 \times 10^{-3}$. All experiments are conducted on Unitree B1-Z1 via an Intel Core i7-14650HX CPU. The implementation is majorly built upon OCS2 \cite{OCS2}.

We evaluate two aspects of the proposed framework. To assess scalability and computational efficiency, experiments are conducted in a non-physical simulator similar to \cite{devincenti2023} by rolling out the system dynamics in (\ref{eq:raw_payload_dynamics}) and (\ref{eq:raw_robot_dynamics}) using only the first-step MPC optimized inputs. The full-body motions are realized through IK without WBC, allowing us to isolate trajectory quality and MPC solving time from tracking errors or lower-level controller effects. To evaluate tracking performance, experiments are performed in a high-fidelity physical simulator with Gazebo physics engine and rendered in Blender. The MPC and WBC run asynchronously to emulate realistic multi-rate interactions between planning and control. For the payload, we use a cargo of length $1\,\mathrm{m}$ in all dimensions with homogeneous density.

% \subsection{Reference Tracking}

\subsection{Rough Terrain}
Figure~\ref{fig:perceptive_terrain_scenarios} summarizes results across rough terrain scenarios. In the Gap case (A.1–A.2), the two-robot team clears stepped terrain with adaptive swing heights while maintaining stable payload transport. In the Slope case (B.1–B.2), the system traverses a $10^\circ$ incline with base and cargo orientation compensating for terrain tilt. The Narrow Turn scenario (C) presents the most constrained configuration, requiring coordinated replanning as the team redirects the payload through a $90 \degree$ turn within a narrow passage. The annular platform scenario (D) demonstrates stable formation tracking under continuous curvature on an elevated circular path.
% Figure~\ref{fig:perceptive_terrain_scenarios} summarizes the quantitative results across all rough terrain scenarios. In the Gap scenario (A.1 and A.2), the two-robot configuration successfully navigates the stepped terrain while maintaining stable payload transport, with the perceptive reference manager adapting swing heights to clear each step. The Slope scenario (B.1 and B.2) demonstrates the two-robot system reliably traverses the $10 \degree$ incline, with the base height trajectory adapting to the terrain gradient and the cargo pitch reference compensating for the surface tilt. The Narrow Turn scenario (C) presents the most constrained configuration, requiring coordinated replanning as the team redirects the payload through a $90 \degree$ turn within a narrow passage. The annular platform scenario (D) further validates the framework under continuous curvature commands, where robots maintain formation while following a circular reference path on the elevated surface.

% For the scalability evaluation, extending the Gap scenario to three robots (E) and the Slope scenario to four robots (F) yields stable transport performance, demonstrating that the proposed ADMM-based coordination scales effectively with team size without significant degradation in trajectory quality or MPC solve time. 

% Across all rough terrain scenarios, the framework maintains stable footing by confining foot placements within valid planar regions and adapting swing trajectories to terrain height variations, validating the effectiveness of the perceptive terrain mode described in Sec.~\ref{sec:approach}.

\subsection{Scalability and Convergence Analysis}\label{subsec:scalability_convergence}
\begin{figure}
    \centering
    \includegraphics[width=\linewidth]{figures/scalability_analysis.png}
    \vspace{-5.0mm}
    \caption{CPU time scalability comparison between distributed and centralized MPC for 2–4 robots in flat and Gap-Slope terrain scenarios. The red dashed line indicates the 30 Hz real-time threshold.}
    \label{fig:scalability_analysis}
    \vspace{-5.0mm}
\end{figure}

\begin{figure}
    \centering
    \includegraphics[width=0.95\linewidth]{figures/mpc_residuals_multi_trial.png}
    \vspace{-2.0mm}
    \caption{Residual convergence (top-left, top-right, and bottom-left) and computation time (bottom-right) for different ADMM-SQP configurations.}
    \label{fig:convergence_analysis}
    \vspace{-4.0mm}
\end{figure}

We use the flat and Gap-Slope (Fig.~\ref{fig:perceptive_terrain_scenarios}-E and F) terrain scenarios as benchmarks for scalability tests between distributed and centralized MPC. The results, shown in Fig. \ref{fig:scalability_analysis}, show computational advantages for distributed MPC that increase with team size. In flat terrain, distributed MPC achieves $3.6 \times$, $7.1 \times$, and $11.4 \times$ speedups for 2, 3, and 4 robots, with median CPU times of $6.73$ ms to $11.63$ ms compared to centralized MPC's $24.38$ ms to $133.13$ ms. In Gap-Slope terrain with perceptive constraints, distributed MPC achieves $1.7 \times$, $5.0 \times$, and $7.3 \times$ speedups with median times of $18.01$ ms, $16.39$ ms, and $23.03$ ms.
% , while centralized MPC reaches 167.95 ms for 4 robots—exceeding a minimally real-time 30 Hz threshold by more than 5×. 
Distributed MPC remains 50 Hz (100 Hz on flat terrain) for all team sizes, whereas centralized MPC is below 30 Hz for 3 and 4 robots. The largest runtime outliers occur during the initial solve due to perception initialization and cold starts.
% , and do not reflect steady-state performance.
The superior scalability of distributed MPC stems from decomposing the multi-robot optimization into smaller, parallelizable subproblems, resulting in nearly uniform computational time across team sizes with minimal overhead from perceptive-related updates and processing, whereas centralized MPC must solve a single, exponentially growing problem that becomes computationally prohibitive as team size increases.

To evaluate ADMM convergence and warm-start effects, we ran 60 trials on a Gap terrain (Fig.~\ref{fig:perceptive_terrain_scenarios}-E) with different ADMM-SQP configurations. Each trial used a waypoint sent across the gap with random offsets of $\pm 1$ m in x-y and $\pm 90\degree$ in yaw. As shown in Fig. \ref{fig:convergence_analysis}, with 1 ADMM iteration, the residual drops more slowly initially than with more iterations, and deviation increases, especially in the gap phase, where the consensus residual becomes unsuppressed. More SQP iterations slightly reduce large constraint violations. With 2 ADMM iterations, the residual stays within the threshold with rare violations; 5 iterations improve further. To capture computational differences, we recorded computation time for the first 0.5 s after MPC starts, since the initial steps must quickly reduce the residual for consensus and the solver starts from scratch. 
Results show that, under the same SQP iteration, more ADMM iterations converge faster initially but are more expensive and show sparser time distributions, as more ADMM iterations are needed to converge, slowing the process initially. More SQP iterations substantially increase computation time under the same ADMM iteration. Therefore, we chose 2 ADMM iterations and 1 SQP iteration to balance constraint satisfaction and computational efficiency.

\subsection{Obstacle Avoidance}


\begin{figure}
    \centering
    \includegraphics[width=\linewidth]{figures/obstacle_overview.png}
    \vspace{-5.0mm}
    \caption{Obstacle avoidance and pose tracking performance. 
    % The robots follow the optimized collision-free trajectory from start to goal. Tracking errors are shown over a representative 10\,s segment of the full 110\,s task, with maximum linear and angular errors of 0.0183\,m and 1.644$^\circ$, respectively.
    }
    \label{fig:obstacle_avoidance_overview}
    \vspace{-5.0mm}
\end{figure}

To evaluate the obstacle avoidance capability of the proposed MPC framework, a scenario was constructed in a physical simulator as shown in Fig.~\ref{fig:perceptive_terrain_scenarios}-G and Fig.~\ref{fig:obstacle_avoidance_overview}. 
% The robot--cargo system starts from an initial position on the left and navigates through multiple box-shaped obstacles to reach a given goal position on the right. Each box obstacle is represented as a spherical region with radius slightly larger than the box size. 
The left portion of Fig.~\ref{fig:obstacle_avoidance_overview} illustrates the resulting trajectory, showing that the robot--cargo system safely traverses all obstacles. 
With only a simple start-to-goal interpolation as reference, the MPC automatically finds a feasible trajectory that turns and leverages manipulator flexibility to pass narrow passages while remaining close to the reference. 
The right portion of Fig. \ref{fig:obstacle_avoidance_overview} shows the pose tracking error over a representative $10$~s segment of the full $110$~s duration, obtained from integrated MPC-WBC. 
% The linear tracking error is computed as the Euclidean norm of the position error, while the angular error is computed using the logarithmic map on $\mathrm{SO}(3)$. 
The maximum linear error is $0.0183$~m, and the maximum angular error is $1.644^\circ$. Periodic spikes in the error correspond to ground impact events during trotting, which introduce transient disturbances but remain bounded. These results demonstrate that the proposed MPC--WBC framework achieves reliable real-time obstacle avoidance while maintaining accurate pose tracking.

\subsection{Robustness and Ablation Study}
% Robustness to cargo mass variation and model uncertainty is a critical performance metric for the proposed controller. 
To evaluate robustness, the robot--cargo system was commanded to track a reference trajectory consisting of a $6.5\,\mathrm{m}$ translation with a simultaneous $90^\circ$ rotation in physical simulations. Multiple trials were performed under different nominal cargo masses as well as under deliberate mass and inertia modeling error. 
As shown in Fig.~\ref{fig:robustness_analysis},
despite rare transient peaks caused by foot--ground impacts, the median errors and interquartile ranges remain consistently low across different nominal masses and under both mass and inertia modeling error, demonstrating strong robustness of the proposed controller. The controller performance degrades to failure when modeling error is approximately $67\%$. 
% Furthermore, the system exhibits greater robustness when the modeled mass or inertia is overestimated compared to underestimated cases, indicating that relatively aggressive parameter estimates provide additional stability margin. This behavior arises because the controller computes larger wrench commands under relatively aggressive estimates, enabling more effective compensation for tracking errors and external disturbances. This phenomenon highlights the effectiveness of explicit wrench computation and tracking within integrated MPC and WBC framework, compared to approaches that rely solely on position-level tracking without explicitly accounting for force consistency.

\begin{figure}[h]
    \vspace{-3.0mm}
    \centering
    \includegraphics[width=0.9\linewidth]{figures/robustness.png}
    \vspace{-2.0mm}
    \caption{Cargo pose tracking errors under mass and inertia variations. 
    % Box plots show the distribution of position and orientation tracking errors for different nominal masses and parameter perturbations.
    }
    \label{fig:robustness_analysis}
    \vspace{-2.0mm}
\end{figure}

Another important feature of our framework is that we optimize and track the full wrench including both force and torque at the grasped handles. As an ablation study, when torque is completely disabled, the angular tracking error increases over time and eventually leads to instability, demonstrating the limitation of force-only tracking. Enabling torque, particularly along the alignment axis between the robot and cargo (x-axis), significantly improves rotational stability and reduces angular error, the quantitive result is shown in Fig. \ref{fig:wrench_ablation}.
% Torque about the z-axis is tightly constrained to prevent slipping at the gripper--handle interface. This anisotropic torque constraint enables stable and accurate tracking while maintaining safe contact interaction.
\begin{figure}[h]
    % \vspace{-2.0mm}
    \centering
    \includegraphics[width=0.9\linewidth]{figures/wrench_ablation.png}
    \vspace{-3.0mm}
    \caption{Comparison between force-only tracking and full wrench tracking.
    % Force-only tracking exhibits large angular error and eventually becomes unstable, while wrench tracking maintains stable behavior.
    }
    \label{fig:wrench_ablation}
    \vspace{-4.0mm}
\end{figure}

% Benchmark metrics: 

% For each ADMM/SQP iteration, set a minimum constraint violation; i.e., how does changing the constraint satisfaction affect the convergence or number of iterations needed? 

% For different manual ADMM/SQP iteration setup, report the constraint satisfaction  

% For verifying the WBC tracking performance. Simulation in gazebo for two-robott carrying task.

% Based the task, we divide them into 1) Obstacle avoidance; 2) Narrow passage; 3) Slope.

\section{Conclusion}
\label{sec:conclusion}
This paper presented an ADMM-based distributed MPC framework for collaborative prehensile loco-manipulation with multiple quadruped manipulators. Exploiting the star-shaped coupling induced by a shared payload, the centralized optimal control problem is decomposed into parallel robot-level subproblems with consensus on interaction wrenches, preserving dynamic coupling while improving scalability. Combined with a wrench-aware whole-body controller, the framework achieves real-time, force-consistent execution. Simulations with up to four robots validate scalability, fast convergence with few ADMM iterations, and robustness to model and terrain variations. Future work includes hardware validation on physical platforms, and exploring GPU implementations to guide decentralized RL training.

% Future work includes implementation with full-order robot dynamics and GPU-accelerated implementation to guide communication-free decentralized RL training.

\bibliographystyle{IEEEtran}
\bibliography{references}

\end{document}
