\section{Related Work}
\label{sec:related_work}
\subsection{Collaborative Transportation and Manipulation}
% Legged robots. In \cite{vincenti_centralized_2023}, a centralized MPC is proposed to compute motion plans for multiple single rigid bodies including the carrying object. The approach is centralized and does not scale up to the number of robots. With passive chain or active actuation.

% Other distributed approach depends on a global navigation planner.

% Outline:
% \begin{itemize}
%     \item General multi-robot collaborative transportation: Drone; mobile manipulator related works. We broadly categorize them into centralized, decentralized, and leader-follower. 
%     \item Collaborative loco-manipulation is relatively less studied due to legged manipulators' high DoFs and additional balancing need. Depending on how the interaction between the payload and robot happens, there are three main approaches: holonomically constrained system, prehensile system, and non-prehensile system.
%     \item Due to its mechanical simplicity and compatibility with other single-robot manipulation task, in this work, we study pre-hensile system, and focus on tackling the scalabilty issue of centralized approach using a distributed approach.
% \end{itemize}

% Multi-robot collaborative transportation and manipulation has been widely studied across diverse robot morphologies, including aerial platforms \cite{chung2018survey} and ground robots such as fixed or mobile base mounted with manipulators. Existing approaches are commonly categorized into centralized, decentralized, and leader–follower paradigms based on the degree of coordination and information sharing among robots. Centralized methods jointly optimize the trajectories and interaction forces of all agents but suffer from limited scalability \cite{nikou2017nonlinear,sun2025agile}. In contrast, decentralized approaches improve scalability by relying on local sensing and communication but sacrifice for the coordination optimality \cite{khatib1996coordination,verginis2018communication,culbertson2018decentralized}. Leader–follower strategies can be viewed as a partially decentralized formulation \cite{farivarnejad2022multirobot}, where followers infer the leader’s intent with or without explicit communication \cite{stilwell1993toward,sugar2002control}.

Multi-robot collaborative transportation and manipulation has been studied across aerial \cite{chung2018survey} and ground platforms with manipulators. Existing methods are typically categorized as centralized, decentralized, or leader–follower. Centralized approaches jointly optimize all agents but scale poorly \cite{nikou2017nonlinear,sun2025agile}. Decentralized methods improve scalability through local sensing and communication at the expense of coordination optimality \cite{khatib1996coordination,verginis2018communication,culbertson2018decentralized}. Leader–follower strategies represent a partially decentralized paradigm, where followers infer the leader’s intent with or without explicit communication \cite{farivarnejad2022multirobot,stilwell1993toward,sugar2002control}.
However, leader–follower approaches do not explicitly address optimal trajectory planning for the robot group, as the resulting behavior is largely determined by the leader’s injected commands.
% However, optimal trajectory planning for the robot group is not addressed directly in leader-follower approaches.

Compared to wheeled or aerial platforms, collaborative loco-manipulation with legged robots remains relatively less explored due to the high degrees of freedom of legged manipulators and the requirement of maintaining dynamic balance during manipulation. Based on how robots physically interact with the payload, collaborative loco-manipulation has been explored under different interaction paradigms.

\textbf{Holonomically constrained systems}, where robots are mechanically linked via cables or rigid connections, have been addressed using decentralized learning-based controllers relying only on local state and relative payload information \cite{pandit2024} and hierarchical architectures combining centralized or distributed planners with decentralized controllers \cite{kim2023,yang2022}. While these methods scale well, their applicability is limited by restrictive mechanical assumptions and reduced interaction flexibility.

\textbf{Prehensile collaborative loco-manipulation}, which enables more flexible interactions through grasping or holding, has primarily been studied using centralized or hierarchical approaches, including centralized MPC with simplified rigid-body models \cite{devincenti2023}, and hierarchical framework combining payload planning and decentralized whole-body control for collision-free transport of large objects \cite{rigo2025hierarchical}. 
Passive-arm designs have also been used to enable leader–follower coordination through intrinsic mechanical impedance \cite{turrisi2024}. Learning-based decentralized approaches have recently demonstrated contact-only lifting and transport without communication by inducing rigid-like coordination through reward shaping \cite{pandit2025multi}, and hierarchical RL has been applied to collaborative pick-and-place tasks \cite{an2025collaborative}. Despite their flexibility, these methods often face scalability limitations due to centralized optimization, strong hierarchical dependencies, or complex reward and curriculum design.
% In \cite{turrisi2024}, a quadruped is equipped with a passive arm structure, enabling leader–follower coordination through intrinsic mechanical impedance. In \cite{an2025collaborative}, a collaborative pick-and-place task is achieved by a hierarchical, decentralized reinforcement learning (RL) framework. 
% Although prehensile methods are more expressive and compatible with single-robot manipulation, their scalability is often limited by centralized optimization, strong hierarchical dependencies, or complicated curriculum and reward design. 

In contrast, \textbf{non-prehensile approaches} typically rely on indirect manipulation such as pushing \cite{feng2025learning, sombolestan2024} or carrying \cite{ji2021reinforcement} without extra mechanical design. Long-horizon pushing \cite{feng2025learning, sombolestan2024} tasks have been achieved by hierarchical planning frameworks that combine high-level safety-critical payload planning with decentralized RL or MPC controller to handle contact uncertainty and obstacles. These methods support long-horizon tasks but lack direct force control, limiting manipulation precision and task diversity.

% Among these paradigms, prehensile collaborative loco-manipulation offers greater flexibility and compatibility with single-robot manipulation tasks, but typically relies on centralized or hierarchical planners. We therefore propose a distributed MPC framework based on alternating optimization, which enables distributed subsystem updates with local communication while retaining much of the centralized solution quality and improving computational scalability.

Motivated by the flexibility of prehensile manipulation and the scalability limitations of existing centralized and hierarchical approaches, we focus on prehensile collaborative loco-manipulation and propose a distributed MPC framework based on alternating optimization, which enables scalable subsystem updates with local communication while retaining much of the centralized solution quality and improving computational scalability.


\subsection{ADMM for Multi-Robot Distributed Control}
% Outline:
% \begin{itemize}
%     \item ADMM for multi-agent; very briefly;  and multi-robot collaborative transporation and manipulation.
%     \item ADMM for multi-robot path finding; the focus is mainly on the inter-robot constraint such as collision avoidance and the dynamics for each robot is simple. 
%     \item ADMM for collaborative manipulation; the system dynamics is tightly coupled by the manipulated object; For each subsystem, the robot state is augmented by the object model and the consensus is achieved. some works propose to reduce the decision variable for each subsystem with only the controls contributed by the single robot. These works focus more on the interaction force and torques but simplify the robot's own kinematics or dynamics.
%     \item In our work, instead of augmenting the local state by the object state, we treat the object as an individual system and reach consensus between the force and torque inputs. This reduces the number of decision variables needed in each subproblem.
% \end{itemize}

% Many optimal control problems (OCPs) exhibit intrinsic distributed structures despite the presence of coupling effects. These structures can be broadly categorized as spatial, temporal, and system-dynamics-level decompositions \cite{zhao2024survey}. In particular, spatial structure arises when subsystems have separate or weakly coupled dynamics, a common characteristic of multi-robot systems. To exploit such structure and improve computational efficiency, these OCPs are often reformulated and solved using alternating optimization methods, most notably the alternating direction method of multipliers (ADMM) \cite{boyd2011distributed}.

% Multi-robot path finding (MAPF) is a representative domain for multi-robot systems where alternating optimization has been successfully applied \cite{saravanos2023distributed,tajbakhsh2025asynchronous}. In these problems, robots typically have independent dynamics and are coupled only through constraints such as collision avoidance. Each agent maintains local copies of its own and neighboring trajectories and enforces consensus over inter-robot constraints. This formulation scales well to large teams, but it often considers simplified robot models, ensuring that each subproblem remains computationally tractable despite the augmented state dimension.

Many OCPs exhibit intrinsic distributed structures despite coupling effects, which can arise at the spatial, temporal, or system-dynamics level \cite{zhao2024survey}. In multi-robot systems, spatial separability enables efficient decomposition via alternating optimization, particularly ADMM \cite{boyd2011distributed}. Multi-robot path finding (MAPF) is an example where these ideas have been successfully applied \cite{saravanos2023distributed,tajbakhsh2025asynchronous}: robots maintain local copies of their own and neighboring agents' trajectories, enforce consensus over coupling constraints such as collision avoidance, and achieve scalability to large teams. However, these formulations mostly consider simplified robot models, which keep subproblem computationally tractable despite the augmented state dimension for each robot.

Collaborative manipulation also exhibits an implicit distributed structure, although the subsystems are coupled through shared object dynamics. 
% Applying MAPF-style decompositions to these problems leads to an augmented robot-object model whose control dimension grows with the number of robots. 
To avoid the growing number of control inputs for the robot-object subsystem, \cite{shorinwa2020scalable} proposes a decomposition in which each subsystem optimizes only the force and torque contributed by a single robot, while consensus is enforced on the object trajectory. A similar idea is extended to contact-implicit manipulation settings in \cite{shorinwa_disco_2024}. 
% While these approaches effectively reduce per-agent decision variables, they largely abstract away robot kinematics and internal dynamics, focusing instead on interaction wrenches.
For collaborative loco-manipulation with legged manipulators, each robot subsystem is already high-dimensional due to multi-contact dynamics and balance constraints. Instead of augmenting robot states with object states, we treat the object as an independent subsystem and couple the systems only through interaction forces and torques, on which consensus is enforced. 
% In contrast, for collaborative loco-manipulation with legged manipulators, the complexity of each individual robot system is already high due to multiple contact points and balance requirements. Rather than augmenting each robot’s state with the object state to fully capture the coupling effects, we treat the object as an independent subsystem and augment each robot’s control input with the corresponding interaction force and torque applied by the object.
% Consensus is then enforced directly on these force and torque variables.
This design keeps each subproblem compact and enables scalable distributed MPC despite the nonlinear and high-dimensional dynamics.
% This formulation further reduces the size of each subproblem and is critical for enabling scalable distributed MPC in the presence of strong nonlinearity and high-dimensional robot dynamics.




% The Alternating Direction Method of Multipliers (ADMM) has emerged as a powerful technique for decomposing complex optimization problems in robotics. Amatucci et al.~\cite{amatucci2022} propose an approach to enhance Model Predictive Control (MPC) for legged robots by leveraging Distributed Optimization with the Alternating Direction Method of Multipliers (ADMM). Their method focuses on decomposing the robot's dynamics into smaller, parallelizable subsystems. Each of these subsystems is then managed by its own Optimal Control Problem (OCP). The core idea is to transform the complex, centralized optimization problem of a whole-body robot into a network of smaller, local independent problems that can be solved in parallel. For instance, a quadruped robot can be decoupled into two parts: front and back. A quadruped with an articulated arm could be split into three subsystems: front, back, and the arm. ADMM ensures consistency between the optimizations of these decoupled subsystems. It facilitates this by enforcing consensus among them, preventing non-physical behaviors. The decomposition allows each subsystem to consider only its relevant decision variables and constraints, significantly reducing the problem size for each local OCP. This distributed approach significantly decreases computational time.
